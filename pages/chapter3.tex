% Chapter 3

\chapter{引用与链接}

\section{脚注}
注释是对论文中特定名词或新名词的注解。注释可用页末注或篇末注的一种。选择页末注的应在注释与正文之间加细线分隔,线宽度为 1 点,线的长度不应超过纸张的三分之一宽度。同一页类列出多个注释的,应根据注释的先后顺序编排序号。字体为宋体5号,注释序号以“\circled{1}、\circled{2}”等数字形式标示在被注释词条的右上角。页末或篇末注释条目的序号应按照“\circled{1}、\circled{2}”等数字形式与被注释词条保持一致。示例:这里有个注释\footnote{我是解释注释的}。

\section{引用文中小节}\label{sec:ref}
如引用小节~\ref{sec:ref}

\section{引用参考文献}
这是一个参考文献引用的范例\cite{kuhn2004man}

还可以采用上标的引用方式\upcite{江泽民1989能源发展趋势及主要节能措施}

引用多个文献\cite{kuhn2004man,江泽民2008新时期我国信息技术产业的发展,江泽民1989能源发展趋势及主要节能措施}

文献引用需要配合 \hologo{BibTeX} 使用,很多工具可以直接生成 \hologo{BibTeX} 文件(如 \app{EndNote}、\app{NoteExpress}、\app{百度学术}、\app{谷歌学术}等),此处不作介绍。

\section{链接相关}
模板使用了 \pkg{hyperref} 包处理相关链接,使用 \verb|\href| 可以生成超链接,链接的颜色只在 \verb|class=manual| 中显示,而在 \verb|paper| 与 \verb|design| 下不显示。如果需要输出网址,可以使用 \verb|\url| 命令,示例:\url{https://github.com}。