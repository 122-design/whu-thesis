% Chapter 2

\chapter{公式插图表格}
\section{公式的使用}
在文中引用公式可以这么写:\(a^2 + b^2 = c^2\)。这是勾股定理,它还可以表示为 \(c = \sqrt{a^2 + b^2}\)。还可以让公式单独一段并且加上编号:
\begin{equation}
  \sin^2{\theta} + \cos^2{\theta} = 1 \label{eq:pingfanghe}
\end{equation}
注意,公式前请不要空行。

还可以通过添加标签在正文中引用公式,如式\eqref{eq:pingfanghe}。

我们还可以轻松打出一个漂亮的矩阵:
\begin{equation}
  \vb*{A} =
  \begin{bmatrix}
    1  & 2  & 3  & 4  \\
    11 & 22 & 33 & 44 \\
  \end{bmatrix} \times
  \begin{bmatrix}
    22 & 24 \\
    32 & 34 \\
    42 & 44 \\
    52 & 54 \\
  \end{bmatrix}
\end{equation}

或者多行对齐的公式:
\begin{equation}
  \begin{aligned}
    f_1(x) & = (x + y)^2         \\
           & = x^2 + 2 x y + y^2
  \end{aligned}
\end{equation}

\begin{notice}
  \WhuThesis 使用了 \pkg{unicode-math} 包更改数学字体。所以在使用数学字体时,尽量使用 \pkg{unicode-math} 包提供的 \verb|\sym| 接口,详情请阅读 \pkg{unicode-math} 文档。
\end{notice}

\section{插图的使用}
\LaTeX{} 环境下可以使用常见的图片格式:\fmt{JPEG}、\fmt{PNG}、\fmt{PDF}、\fmt{EPS} 等。当然也可以使用 \LaTeX{} 直接绘制矢量图形,可以参考 \pkg{pgf}/\pkg{ti\textit{k}z}等包中的相关内容。需要注意的是,无论采用什么方式绘制图形,首先考虑的是图片的清晰程度以及图片的可理解性,过于不清晰的图片将可能会浪费很多时间。

\begin{figure}[!htb]
  \centering
  \includegraphics[width=0.3\textwidth]{figures/whulogo.pdf}
  \caption{插图示例}
  \label{fig:whu}
\end{figure}

\verb|[htbp]| 选项分别是此处、页顶、页底、独立一页。\verb|[width=\textwidth]| 让图片占满整行,或 \verb|[width=2cm]| 直接设置宽度。可以随时在文中进行引用,如图~\ref{fig:whu},建议缩放时保持图像的宽高比不变。

\section{表格的使用}
表格的输入可能会比较麻烦,可以使用在线的工具,如 \href{https://www.tablesgenerator.com/}{\app{Tables Generator}} 能便捷地创建表格,也可以使用离线的工具,如 \href{https://ctan.org/pkg/excel2latex}{\app{Excel2LaTeX}} 支持从 Excel 表格转换成 \LaTeX{} 表格。\href{https://en.wikibooks.org/wiki/LaTeX/Tables}{LaTeX/Tables} 上及 \href{https://www.tug.org/pracjourn/2007-1/mori/mori.pdf}{Tables in LaTeX} 也有更多的示例能够参考。

\subsection{普通表格}
下面是一些普通表格的示例:

\begin{table}[ht]
  \centering
  \caption{简单表格}
  \label{tab:1}
  \begin{tabular}{|l|c|r|}
    \hline
    我是 & 一只 & 普通 \\
    \hline
    的   & 表格 & 呀   \\
    \hline
  \end{tabular}
\end{table}

也可以使用 \pkg{booktabs} 包创建三线表。

\begin{table}[ht]
  \centering
  \caption{一般三线表}
  \label{tab:2}
  \begin{tabular}{ccc}
    \toprule
    姓名 & 学号 & 性别 \\
    \midrule
    张三 & 001  & 男   \\
    李四 & 002  & 女   \\
    \bottomrule
  \end{tabular}
\end{table}

三线表中三条横线分别使用 \verb|\toprule|、\verb|\midrule| 与 \verb|\bottomrule|。另可使用 \verb|\cmidrule{m-n}| 添加 \(m\)--\(n\) 列的横线线。

\begin{notice}
  使用三线表时,请牢记
  \begin{enumerate}
    \item 永远不要加竖线
    \item 不要使用双横线
  \end{enumerate}
\end{notice}

要创建占满整个文字宽度的表格需要使用到 \pkg{tabularx} 包提供的 \env{tabularx}环境。引用表格与其它引用一样,只需要表~\ref{tab:3}。

\begin{table}[ht]
  \centering
  \caption{占满文字宽度的三线表}
  \label{tab:3}
  \begin{tabularx}{\textwidth}{CCCC}
    \toprule
    序号 & 年龄 & 身高   & 体重  \\
    \midrule
    1    & 14   & 156    & 42    \\
    2    & 16   & 158    & 45    \\
    3    & 14   & 162    & 48    \\
    4    & 15   & 163    & 50    \\
    \cmidrule{2-4} %添加2-4列的中线
    平均 & 15   & 159.75 & 46.25 \\
    \bottomrule
  \end{tabularx}
\end{table}

\subsection{跨页表格}
跨页表格常用于附录(把正文懒得放下的实验数据统统放在附录的表中)。一般使用 \pkg{longtable} 包提供的 \env{longtable} 环境。若要要创建占满整个文字宽度的跨页表格,可以使用 \pkg{xltabular} 提供的 \env{xltabular} 环境,使用方法与 \pkg{longtable} 类似。以下是一个文字宽度的跨页表格的示例:

\begin{xltabular}{\textwidth}{CCCCCCCCC}
  \caption{文字宽度的跨页表格示例}  \\
  \toprule
  1 & 0 & 5 & 1 & 2 & 3 & 4 & 5 & 6 \\
  \midrule
  \endfirsthead

  \multicolumn{9}{l}{接上一页}      \\
  \toprule
  1 & 0 & 5 & 1 & 2 & 3 & 4 & 5 & 6 \\
  \midrule
  \endhead

  \toprule
  \multicolumn{9}{r}{转下一页}
  \endfoot

  \bottomrule
  \endlastfoot

  1 & 0 & 5 & 1 & 2 & 3 & 4 & 5 & 6 \\
  1 & 0 & 5 & 1 & 2 & 3 & 4 & 5 & 6 \\
  1 & 0 & 5 & 1 & 2 & 3 & 4 & 5 & 6 \\
  1 & 0 & 5 & 1 & 2 & 3 & 4 & 5 & 6 \\
  1 & 0 & 5 & 1 & 2 & 3 & 4 & 5 & 6 \\
  1 & 0 & 5 & 1 & 2 & 3 & 4 & 5 & 6 \\
  1 & 0 & 5 & 1 & 2 & 3 & 4 & 5 & 6 \\
  1 & 0 & 5 & 1 & 2 & 3 & 4 & 5 & 6 \\
  1 & 0 & 5 & 1 & 2 & 3 & 4 & 5 & 6 \\
  1 & 0 & 5 & 1 & 2 & 3 & 4 & 5 & 6 \\
  1 & 0 & 5 & 1 & 2 & 3 & 4 & 5 & 6 \\
  1 & 0 & 5 & 1 & 2 & 3 & 4 & 5 & 6 \\
  1 & 0 & 5 & 1 & 2 & 3 & 4 & 5 & 6 \\
  1 & 0 & 5 & 1 & 2 & 3 & 4 & 5 & 6 \\
  1 & 0 & 5 & 1 & 2 & 3 & 4 & 5 & 6 \\
  1 & 0 & 5 & 1 & 2 & 3 & 4 & 5 & 6 \\
  1 & 0 & 5 & 1 & 2 & 3 & 4 & 5 & 6 \\
  1 & 0 & 5 & 1 & 2 & 3 & 4 & 5 & 6 \\
  1 & 0 & 5 & 1 & 2 & 3 & 4 & 5 & 6 \\
  1 & 0 & 5 & 1 & 2 & 3 & 4 & 5 & 6 \\
  1 & 0 & 5 & 1 & 2 & 3 & 4 & 5 & 6 \\
  1 & 0 & 5 & 1 & 2 & 3 & 4 & 5 & 6 \\
  1 & 0 & 5 & 1 & 2 & 3 & 4 & 5 & 6 \\
  1 & 0 & 5 & 1 & 2 & 3 & 4 & 5 & 6 \\
  1 & 0 & 5 & 1 & 2 & 3 & 4 & 5 & 6 \\
  1 & 0 & 5 & 1 & 2 & 3 & 4 & 5 & 6 \\
  1 & 0 & 5 & 1 & 2 & 3 & 4 & 5 & 6 \\
  1 & 0 & 5 & 1 & 2 & 3 & 4 & 5 & 6 \\
  1 & 0 & 5 & 1 & 2 & 3 & 4 & 5 & 6 \\
  1 & 0 & 5 & 1 & 2 & 3 & 4 & 5 & 6 \\
  1 & 0 & 5 & 1 & 2 & 3 & 4 & 5 & 6 \\
  1 & 0 & 5 & 1 & 2 & 3 & 4 & 5 & 6 \\
\end{xltabular}


\section{列表的使用}
下面演示了创建有序及无序列表,如需其它样式,\href{https://www.latex-tutorial.com/tutorials/lists/}{LaTeX Lists} 上有更多的示例。

\subsection{有序列表}
这是一个计数的列表
\begin{enumerate}
  \item 第一项
        \begin{enumerate}
          \item 第一项中的第一项
          \item 第一项中的第二项
        \end{enumerate}
  \item 第二项
        \begin{enumerate}[label=(\roman*)]
          \item 第一项中的第一项
          \item 第一项中的第二项
        \end{enumerate}
  \item 第三项
\end{enumerate}

\subsection{不计数列表}
这是一个不计数的列表
\begin{itemize}
  \item 第一项
        \begin{itemize}
          \item 第一项中的第一项
          \item 第一项中的第二项
        \end{itemize}
  \item 第二项
  \item 第三项
\end{itemize}

\section{数学环境的使用}
\WhuThesis 简单定义了 8 种数学环境,具体见表~\ref{tab:数学环境},使用方法如下所示。

\begin{table}
  \caption{\WhuThesis 定义的数学环境}\label{tab:数学环境}
  \begin{tabularx}{\textwidth}{CCCC}
    \toprule
    theorem     & definition & lemma  & corollary \\
    定理        & 定义       & 引理   & 推论      \\
    \midrule
    proposition & example    & remark & proof     \\
    性质        & 例         & 注     & 证明      \\
    \bottomrule
  \end{tabularx}
\end{table}

\begin{theorem}
  设向量 \(\vb*{a} \neq \vb*{0}\),那么向量\(\vb*{b} \parallel \vb*{a}\) 的充分必要条件是:存在唯一的实数 \(\lambda\),使 \(\vb*{b} = \lambda \vb*{a}\)。
\end{theorem}
\begin{definition}
  这是一条定义。
\end{definition}
\begin{lemma}
  这是一条引理。
\end{lemma}
\begin{corollary}
  对数轴上任意一点 \(P\),轴上有向线段 \(\overrightarrow{OP}\) 都可唯一地表示为点 \(P\) 的坐标与轴上单位向量 \(\vb*{e}_u\) 的乘积:\(\overrightarrow{OP} = u \vb*{e}_u\)。
\end{corollary}
\begin{proposition}
  这是一条性质。
\end{proposition}
\begin{example}
  这是一条例。
\end{example}
\begin{remark}
  这是一条注。
\end{remark}
\begin{proof}
  留作练习。
\end{proof}

若要定义自己的数学环境,可通过如下代码实现:
\begin{lstlisting}[language=tex]
\newtheorem{nonsense}{胡说}
\newtheorem*{bullshit}{八道}
\end{lstlisting}
其中,带星号 * 的命令不会自动编号。

\newtheorem{nonsense}{胡说}
\newtheorem*{bullshit}{八道}

\begin{nonsense}
  啊吧啊吧啊吧。
\end{nonsense}

\begin{bullshit}
  不啦不啦不啦。
\end{bullshit}